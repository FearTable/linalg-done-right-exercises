\renewcommand*{\thechapter}{Chapter \arabic{chapter}}
%\renewcommand*{\thesection}{\Roman{section}}
\renewcommand*{\thesection}{\arabic{chapter}\Alph{section}}
\renewcommand*{\thesubsection}{\arabic{chapter}\Alph{section}}

\newcommand{\gt}{>}
\newcommand{\lt}{<}

%\def\deltaLow{{ \color{purple}{ \delta_{\text{low}} }}}
%\def\deltaSup{{ \color{purple}{ \delta_{\text{sup}} }}}
%\def\deltaHigh{{ \color{purple}{ \delta_{\text{high}} }}}
%\def\deltaVanilla{{ \color{purple}{ \delta }}}
%
%\def\tHigh{{ \color{olive}{ t_{\text{high}} }}}
%\def\tLow{{ \color{olive}{ t_{\text{low}} }}}
%
%\def\supremum{{ \color{RoyalBlue}{ \tilde s }}}
%\def\SSet{{ \color{RoyalBlue}{ S }}}
%
%\def\myEpsilon{{ \color{black}{\epsilon }}}
%\def\myspan{{ \color{black}{\text{span} }}}


\newcommand\myand{\;\;\, \text{and}\;\;}
\newcommand\myor{\;\;\, \text{or}\;\;}
\newcommand\dx{\, dx}
\newcommand\dy{\, dy}
\newcommand\dz{\, dz}
\newcommand\du{\, du}

\def\degree{{\operatorname{deg} \,}}

\newcommand{\myspan}[1]{\operatorname{span} (#1)}

%Fast way to write v_1 ... v_n
\newcommand{\oneTillN}[1]{#1_1, \dots, #1_n}
\newcommand{\onetilln}[1]{#1_1, \dots, #1_n}

%Fast way to write v_1 ... v_m
\newcommand{\oneTillM}[1]{#1_1, \dots, #1_m}
\newcommand{\onetillm}[1]{#1_1, \dots, #1_m}

% fast way to write v_1 ... v_{#2}
% usage \onetill{v}{k-1} yields v_1 \dots v_{k-1}
\newcommand{\oneTill}[2]{#1_1, \dots, #1_{#2}}
\newcommand{\onetill}[2]{#1_1, \dots, #1_{#2}}

\newcommand{\kInOneTillM}{k \in \{1, \dots, m \}}
\newcommand{\kinonetillm}{k \in \{1, \dots, m \}}
\newcommand{\kInOneTillN}{k \in \{1, \dots, n \}}
\newcommand{\kinonetilln}{k \in \{1, \dots, n \}}
\newcommand{\kInOneTillP}{k \in \{1, \dots, p \}}
\newcommand{\kinonetillp}{k \in \{1, \dots, p \}}

% abreviation for finite-dimensional vector space
\newcommand{\findimvecpac}{finite-dimensional vector space }
\newcommand{\findimvs}{finite-dimensional vector space }
\newcommand{\fdvs}{finite-dimensional vector space }

%abreviation for linearly independent
\newcommand{\lid}{linearly independent }

%abreviation for linearly independent
\newcommand{\ld}{linearly dependent }

%abreviation for linearly independent
\newcommand{\vs}{vector space }

%abreviation for finite-dimensional
\newcommand{\fd}{{finite-dimensional }}

%abreviation for linear map
\newcommand{\lm}{{linear map }}

%abbreaviation for L(V,W)
\newcommand{\lvw}{{\mathcal{L}(V,W)}}

\newcommand{\linmap}{\mathcal{L}}
\newcommand{\lin}[2]{{\mathcal{L}(#1, #2)}}

\newcommand{\mynull}{\operatorname{null}}

\newcommand{\myrange}{\operatorname{range}}

\newcommand{\even}{\operatorname{even}}
\newcommand{\odd}{\operatorname{odd}}

%\newcommand{\mmatrix}{\mathcal{M}}

% Natural numbers, integers, real numbers, complex numbers:
\newcommand{\nat}{\mathbb{N}}
\newcommand{\integer}{\mathbb{N}}
\newcommand{\real}{\mathbb{R}}
\newcommand{\compl}{\mathbb{C}}
\newcommand{\myF}{\mathbb{F}}

% Polynomial symbol:
\newcommand{\polyn}{\mathcal{P}}

% Matrix symbol:
\newcommand{\mmatrix}{\mathcal{M}}

%\newcommand{\bfemph}[1]{{\ttfamily\fontseries{b}\selectfont #1}}
\newcommand{\bfemph}[1]{{\fontseries{b}\selectfont #1}}

%\newcommand{\basis}[2]{\overbrace{ \myspan{#1_1, \dots #1_{#2}}}^{\text{linearly independent}} }}

\def\myimpl{{ \color{black}{\implies}}}


\def\bold#1{{\bf #1}}

\newtheoremstyle{mytheoremstyle} % name
%{\topsep}                    % Space above
{0.7em}                    % Space above
{0em}                        % Space below
{}                           % Body font
{0em}                           % Indent amount
%{\ttfamily\fontseries{b}\selectfont}                   % Theorem head font
{\fontseries{sb}\selectfont}                   % Theorem head font
{:\newline}                          % Punctuation after theorem head
{.3em}                       % Space after theorem head
{}  					     % Theorem head spec (can be left empty, meaning ‘normal’)

%\theoremstyle{mytheoremstyle}
%\newtheorem{thm}{Thm}[chapter]
%
%\theoremstyle{mytheoremstyle}
%\newtheorem{mydef}{Def}[chapter]
%
%\theoremstyle{mytheoremstyle}
%\newtheorem{Example}{Example}[chapter]


%\theoremstyle{mytheoremstyle}
%\newtheorem{thm}{Thm}[chapter]
%
%\theoremstyle{mytheoremstyle}
%\newtheorem{mydef}[thm]{Def}
%
%\theoremstyle{mytheoremstyle}
%\newtheorem{example}[thm]{Example}

\theoremstyle{mytheoremstyle}
\newtheorem{thm}{Theorem}[chapter]

\theoremstyle{mytheoremstyle}
\newtheorem{mydef}[thm]{Definition}
\newtheorem*{mydef-non}{Definition}

\theoremstyle{mytheoremstyle}
\newtheorem{example}[thm]{Example}

\setlength{\abovedisplayskip}{1pt}
\setlength{\belowdisplayskip}{1pt}
